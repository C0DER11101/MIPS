\documentclass{article}
\usepackage{graphicx}
\usepackage{hyperref}
\usepackage{moreverb}

\makeatletter

\renewcommand\paragraph{\@startsection{paragraph}{4}{\z@}{-3.25ex \@plus -1ex \@minus -.2ex}{1.5ex \@plus .2ex}{\normalfont\normalsize\bfseries}}

\makeatother

\begin{document}

\section{The Bitmap Display}

\paragraph{}
This is the picture of the bitmap display in MIPS.

\begin{figure}[h]
\centering
\includegraphics[width=0.9\textwidth]{bitmapDisplay.png}  % width=0.9 means 90% of the text width
\caption{The MIPS bitmap display}
\label{fig:example}
\end{figure}

There are various options in this display as seen from the image. All these options can be adjusted to the desired values.

There is a button called \textbf{Connect to MIPS} in the bitmap display, before we run/assemble our MIPS code, we need to connect our bitmap display to our MIPS IDE. Then when we run our program, we will be able to see the output of our program on this display(if the code was written to work with the bitmap display).

\paragraph{Code}

View the code \href{https://github.com/C0DER11101/MIPS/blob/MIPS/Project/simple/simple1.asm}{here}.

\begin{verbatimtab}[4]
.data
	row: .word 256
	col: .word 512
	frame: .word 0x20000
	.eqv DATA_SIZE 4
\end{verbatimtab}

\textbf{0x20000} represents the decimal number $1,31,072$ which is a product of $256$ and $512$.

\begin{verbatimtab}[4]
li $t0, 100
li $t1, 200
\end{verbatimtab}

Here we are basically taking the $X$ and the $Y$ coordinates for the pixel.

\begin{verbatimtab}[4]
li $t2, 0x0040F7BC
\end{verbatimtab}

Loading a bluish-green color onto the register \textbf{\$t2}

\paragraph{Calculating the memory address to store the pixel into the frame}

For this purpose, we have gone for the row-major approach and used this formula:
$$
addr\ =\ baseAddr\ +\ (rowIndex\ *\ colSize\ +\ colIndex)\ *\ dataSize
$$

\begin{verbatimtab}[4]
mul $t3, $t0, $t7
add $t3, $t3, $t1
mul $t3, $t3, DATA_SIZE
add $t3, $t3, $v1
\end{verbatimtab}

These instructions are basically implementing the above formula.

\begin{verbatimtab}[4]
sw $t2, ($t3)
\end{verbatimtab}

Here, we are storing the pixel into the calculated memory address for the pixel.

\paragraph{Execution}

\begin{figure}[h]
\centering
\includegraphics[width=0.9\textwidth]{bitmapDisplayMenu.png}
\caption{Bitmap display option on MIPS tools menu}
\label{fig:example}
\end{figure}

Click on \textbf{Bitmap Display} option. After that a bitmap window will be shown.

First, assemble the MIPS code using \textbf{F3}. After that press \textbf{F5} to run the code.

After that, a small white dot will be visible on the bitmap display(see image below).

\begin{figure}[h]
\centering
\includegraphics[width=1.2\textwidth]{bitmapDisplayPixel.png}
\caption{A tiny dot on the bitmap display}
\label{fig:example}
\end{figure}

\end{document}