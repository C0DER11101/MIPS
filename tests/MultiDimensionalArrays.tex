\documentclass{article}

\makeatletter

\renewcommand\paragraph{\@startsection{paragraph}{4}{\z@}{-3.25ex \@plus -1ex \@minus -.2ex}{1.5ex \@plus .2ex}{\normalfont\normalsize\bfseries}}

\makeatother

\begin{document}
\section{Multi-dimensional arrays in MIPS}
\subsection{Implementing multi-dimensional arrays}
\paragraph{Ways of implementing multi-dimensional arrays}
\begin{itemize}
\item Row major $\rightarrow$ Most widely used.
\item Column major $\rightarrow$ Not widely used.
\end{itemize}

\subsubsection{Row-major way}
\subparagraph{}

Consider this 2D array:

\begin{center}
int array[3][4];
\end{center}
This is how the array looks like:

\[
\left[
\begin{array}{cccc}
array[0][0] & array[0][1] & array[0][2] & array[0][3] \\
array[1][0] & array[1][1] & array[1][2] & array[1][3] \\
array[2][0] & array[2][1] & array[2][2] & array[2][3] \\
\end{array}
\right]
\]

This is how the array will look like in row-major form:


\begin{table}[h]
\centering
\begin{tabular}{|c|c|}
\hline
0 & array[0][0] \\
\hline
1 & array[0][1] \\
\hline
2 & array[0][2] \\
\hline
3 & array[0][3] \\
\hline
4 & array[1][0] \\
\hline
5 & array[1][1] \\
\hline
6 & array[1][2] \\
\hline
7 & array[1][3] \\
\hline
8 & array[2][0] \\
\hline
9 & array[2][1] \\
\hline
10 & array[2][2] \\
\hline
11 & array[2][3] \\
\hline
\end{tabular}
\caption{Row-Major representation of matrix array}
\label{tab:mytable}
\end{table}

To access the values in the array, we will use the following formula:

$$
addr\ =\ baseAddr\ +\ (rowIndex\ *\ colSize\ +\ colIndex)\ *\ dataSize
$$

\newpage
\subsubsection{Column-major way}

\subparagraph{}

To access the values in the array, we will use the following formula:

$$
addr\ =\ baseAddr\ +\ (colIndex\ *\ rowSize\ +\ rowIndex)\ *\ dataSize
$$

The array in column-representation:

\begin{table}[h]
\centering
\begin{tabular}{|c|c|}
\hline
0 & array[0][0] \\
\hline
1 & array[1][0] \\
\hline
2 & array[2][0] \\
\hline
3 & array[0][1] \\
\hline
4 & array[1][1] \\
\hline
5 & array[2][1] \\
\hline
6 & array[0][2] \\
\hline
7 & array[1][2] \\
\hline
8 & array[2][2] \\
\hline
9 & array[0][3] \\
\hline
10 & array[1][3] \\
\hline
11 & array[2][3] \\
\hline
\end{tabular}
\caption{Column-Major representation of matrix array}
\label{tab:mytable}
\end{table}

\end{document}