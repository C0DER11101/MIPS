\documentclass{article}
\usepackage{hyperref}
\usepackage{moreverb}
\usepackage{graphicx}
\usepackage{varwidth}

\makeatletter

\renewcommand\paragraph{\@startsection{paragraph}{4}{\z@}{-3.25ex \@plus -1ex \@minus -.2ex}{1.5ex \@plus .2ex}{\normalfont\normalsize\bfseries}}

\makeatother

\begin{document}
\section{Displaying a line on the bitmap display}

\paragraph{CODE}

View the code from \href{https://github.com/C0DER11101/MIPS/blob/MIPS/Project/simple/simple2.asm}{here}

\begin{verbatimtab}[4]
.data
	row: .word 256
	col: .word 512
	frame: .space 0x20000
	
	.eqv DATA_SIZE 4
\end{verbatimtab}

\begin{itemize}
\item \textbf{row} basically represents the height of the display. More the number of rows, more is the height.
\item \textbf{col} represents the width of the display. More the number of columns, more is the width.
\end{itemize}

\section{The \textit{drawLine} procedure}
\paragraph{Snippets}

\begin{verbatimtab}[4]
li $t0, 100
li $t1, 200
li $t2, 0x0047FAAF
\end{verbatimtab}

\begin{itemize}
\item \textbf{\$t0} is our $X$-coordinate.
\item \textbf{\$t1} is our $Y$-coordinate.
\item \textbf{\$t2} is our pixel.
\end{itemize}

Now, we want a horizontal line, therefore we will have to move horizontally or in a row.

\noindent
\fbox{\begin{varwidth}{\textwidth}
\textbf{QUICK NOTE:} \underline{\textit{A slight confusion}}
\vspace{10pt}

I often get confused with rows and columns when it comes to working coordinates. As long as it's matrix, I am fine. But this coordinate system gets me confused. I end up getting stuck with which one represents the row and which one represents the column.

\vspace{10pt}

Now, \textbf{\$t0} is our $X$-coordinate which has a value of $100$. Now, an $X$-coordinate moves horizontally(that is in a row, but from one column to another). So, $X$-coordinate represents, in our case, the $100^{th}$ column. In general, \textbf{if $x\ =\ m$ then it basically represents the $m^{th}$ column(in terms of display).}

\vspace{10pt}

Next, \textbf{\$t1} is our $Y$-coordinate which has a value of $200$. A $Y$-coordinate moves vertically(that is in a column, but from one row to another).
So, $Y$-coordiante represents, in our case, the $200^{th}$ row. In general, \textbf{if $y\ =\ n$ then it basically represents the $n^{th}$ row(in terms of display).}
\end{varwidth}}

More snippets:

\begin{verbatimtab}[4]
lw $t6, row
lw $t7, col
\end{verbatimtab}

\begin{itemize}
\item \textbf{\$t6} is the row size(height) of the display.
\item \textbf{\$t7} is the column size(width) of the display.
\end{itemize}

\subsection{The \textit{drawLoop}}
\paragraph{}
Now, since we will move horizontally i.e. in a row, we will use this formula:
$$
addr\ =\ baseAddr\ +\ (rowIndex\ *\ colSize\ +\ colIndex)\ *\ dataSize
$$
\begin{verbatimtab}[4]
mul $t3, $t1, $t7
add $t3, $t3, $t0
mul $t3, $t3, DATA_SIZE
add $t3, $t3, $a0
\end{verbatimtab}

These $4$ instructions are actually implementing the above mentioned formula.

This loop repeats until \textbf{\$t0} becomes equal or greater than 200.
\newpage
After this loop executes $100$ times, we get the following output:

\begin{figure}[h]
\centering
\includegraphics[width=1.2\textwidth]{bitmapDisplayLine.png}
\caption{A line in bitmap display}
\label{fig:example}
\end{figure}


\end{document}